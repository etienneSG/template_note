% Exemple d'utilisation de la classe note

\documentclass{note}

\graphicspath{{images/}}

\begin{document}


% Le titre du papier
\title{Template pour écrire une note}

% Les auteurs et leur numéro d'affiliation
\author{Etienne de Saint Germain\inst{1}\inst{2}}

% Les affiliations (par ordre croissant des numéros d'affiliation) séparées par \and
\institute
{
  Argon Consulting, 122 Rue Edouard Vaillant, 92300 Levallois-Perret\\
  \email{etienne.de-saint-germain@argon-consult.com}
  \and
  CERMICS, Cité Descartes, 6-8 Avenue Blaise Pascal, 77455 Champs-sur-Marne\\
  \email{etienne.gaillard-de-saint-germain@cermics.enpc.fr}
}

% Date de création
\CreationDate{15 novembre 2015}

% Résumé
\abstract{Ceci est le résumé de la note}

% Les mots-clés
\keywords{bouh, cool}

% Création de la page de titre
\maketitle
\thispagestyle{empty}


\section{Comment faire une note ?}

Il suffit de savoir écrire en LaTeX et d'avoir des choses à raconter. (Contrairement à ce que je fais actuellement.)


\section{Une autre section}

Pour caser une référence \cite{UneRef}.


\begin{figure}
  \centering
  \includegraphics[]{images/gnuplot_output/PP_deterministic_T3_R3.produced_quantities.tikz}
  %\input{images/PP_deterministic_T3_R3.produced_quantities}
  \caption{Image gnuplot}
  \label{fig: Gnuplot}
\end{figure}


% La bibliographie

\bibliographystyle{apalike}
\bibliography{template_note}



\end{document}
